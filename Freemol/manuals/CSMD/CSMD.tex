\documentclass[a4paper,12pt]{article}

%# This File is part of the Freemol package
%# (C) 2003 Fabio Mariotti, <fabio.mariotti@scriptsforscience.org>
%# Please read the full copyright statment in the COPYING file

\newcommand\brint[3]{{\left\langle #1\vert #2 \vert #3\right\rangle}}

\begin{document}
\title{The program for Continuous Symmetry Measure: CSM}
\author{F.Mariotti}
\date{}
\maketitle

\begin{abstract}

This paper is the manual for the program CSMD.

\end{abstract}

\tableofcontents

\section{Introduction}

In this paper we introduce a new approach to Continuous Symmetry Measure.
We will focuse more on chirality measures. Even if the method is not yet
a closed and complete procedure we show I we can proposed it as Chirality
Measure. We present results, comparisons with other methodologies and
finally some improvments.

The first section is a short review of other moderns approaches.

The second section introduce our method.
The third section focuse on comparisons with the previously described approach.

The forth one presents some results and comapre them to previous
approaches.

Finally we ficuse on weekness and improvment of the method and chiral measure.

\section{A review of previous computational approaches}

We will focuse mainly on three different approaches which will name as:
the point approach (Avnir and coworker), the symmetry operator expectation value
(Grimme), and a geometric genralised measure (Solomon). Elder works have
their references in these previous papers.

\subsection{Geometrical points approach}

A modern (computationally inexpensive) approach to the
continuous symmetry measure has been given by Avnir
and coworker\cite{Avnir1,Avnir2}. We will shortly describe the method
while the last works are able to describe on their own the usefullness
of this approach: simple molecule like benzene\cite{AvnirBenzene}
as well complex sistems like HIV\cite{AvnirHIV} have been handled
succesfully.

In this case a molecule is replaced simply by points in a three dimensional
space. In first instance each atom will be replaced by a simple point.
The new obtained object will be the subject of our CSM. The CSM will
be described as an euleran distance from a given (nearest) symmetrized
object. In this case we will have a set of points $\{ p_i \}$ describing
the original object and a set of points $\{ P_i \}$ describing the nearest
symmetric object. The CSM will be given by the equation:

\begin{equation}
CSM \propto \sum_i \frac1N \left( p_i - P_i \right)^2
\end{equation}

A specialized procedure to obtain the set of $P$ points is
described in the Avnir paper\cite{AvnirPoints}. The normalization constant
depend on the chosen procedure to obtain the set of $P$ points
and on the the scale we would like to project the CSM.

In other papers Avnir and coworker extend the work to objects
with a given uncertainty??\cite{AvnirDist} and they apply as well
the method in fields different from chemistry\cite{AvnirGraphic}.

The main ponits of the procedure are: the construction of the symmetrized object; The kind of distance that is considered; The semplicfications involved.

\subsection{The symmetry operation eigenvalue}

A complete different approach has been proposed by Grimme\cite{GrimmeCSM}.
The new idea is to use the concept of eigenvalue as CSM.
Given an eigenstate $\Psi$ we can try to get the eigenvalue of
a given symmetry operation:

\begin{equation}
CSM_S \propto \brint{\Psi}{S}{\Psi}
\end{equation}

This definition is an attempt to introduce quntum mechanical values into
a geometrical idea. This idea is worth to be explored. Never the less
a lower level of chirality measure is already proposed by \cite{quack:chir1}
at hamiltonian level.

What in principle is hidden in the Grimme procedure is the definition of
a symmetric object. We can show that this definition require a symmetric
object or at least is equivalent to a formulation which include a symmetrized
object.

\subsubsection{The existence of symmetrized object}

GET IT FROM OTHER WRITES

\subsection{Geometric Genralised Measure (Solomon)}

This work try to describe the chirality (extension to symmetry in general
will need a harder work) as a how property of the considered object.
WE can demostrate again that the independence from a given symmetrized object
is again a question of formalism. In this case the work of solom is more
mathematically demanding and a description of his work should be readed
from his paper \cite{solomon:chirality}.
Havinf his paper at hand we whould like to note that the definition of
the $\Phi'$ object which is builted from the specular symmtric image
of half of the object itself is nothing but a definition of symmetrized
object. In this case we should speculate as well to the absence of an
optimization procedure of the measure itself.

It could be proved in some cases that that procedure gives the better
measure but does not exist a general prove.

\section{CSM density based}

Beside speculation on other methods, the only thing we can say is: We where
not able to find a procedure which decribe the chirality without getting
into consideration a symmetrized or specular object. We have no mathematical
proves but it sounds quite ragionable that at least we have to compare
the object with it's own specular image.

The primitive definition of chirality is consistent with this feeling:
A chiral object is such when we cannot superimpose the object itself with
it's mirror image.

We need at least the mirror image.

In our approach we define first a symmetrized object. We would like to
extend the problem to symmetry in general and to see the chirality problem
as a special case: the absence or measure of absence of a mirror plane.

In our case we focuse on the electron density as the object at hand.
The procedure can obviously extended to other functions and problems.
Given a density $\rho(r)$ we define the simmetrized object as:

\begin{equation}
\tilde\rho
\end{equation}

\section{Preparing for tests}

\subsection{Some octave programs}

We use gaussians on each nuclei. So we need to define the
gaussian parameters which are fitting the density.

For tests we use approximated gaussians generated by other
parameters. We use the atomic radii to compute the zeta values
for the gaussians. The gaussian coefficients are generated
by imposing a given amount of charge on the atom.

We use the matlab script \verb|genGTOcoeff| to get all the
data for each atom.

Running it as \verb|genGTOcoeff(1,0,ard(1))| we will built
a gaussain function for a neutral hydrogen. The parameters
are respectively the nuclei charge, the atomic charge and
the atimic radio. The radio comes from a prebuilted vector
of atomic radii.

\begin{verbatim}
octave:7> [norm, zeta] = genGTOcoeff(1,0,ard(1))
norm = 3.8384
zeta = 7.7016
octave:8> 
\end{verbatim}

This function computes the value with this code:

\begin{verbatim}
function [gnorm, zeta] = genGTOcoeff(nc,ac,ar)
#
# Produce the exponent and normalization
# for a given input of nuclear charge
# atomic charge and atomic radii
#
# nc = input("Insert the Nuclei Charge: ");
# ac = input("Insert the Atomic Charge: ");
# ar = input(" Insert the Atomic Radii: ");
#
# From the atomic radii we generate the zeta
#
zeta = - log(0.5)/(ar*ar);
#
# From the integration to fit the number of electrons
# we get the gto coefficient. numelec = nc - ac
#
ne = nc - ac;
global gto_zeta;
gto_zeta = zeta;
#[val ier nfun err] = quad("gtorz",0,Inf);
val = quad("gtorz",0,Inf);
gnorm = ne/val;
#
endfunction
\end{verbatim}

The \verb|zeta| is choosen to generate a gaussian with a value at half height
to match the atomic radii (a gaussian centered on the origin).

The \verb|norm| normalization factor is choosen to generate a gaussian
that after integration gives the number of electrons.

\section{Computed results}

\section{Conclusions}


\begin{thebibliography}{1}

\end{thebibliography}



\end{document}
\end{article}

